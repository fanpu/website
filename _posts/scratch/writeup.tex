% You should title the file with a .tex extension (hw1.tex, for example)
\documentclass[11pt]{article}
% \usepackage{tikz}
\usepackage{amsmath}
\usepackage{listings} %For code in appendix
\lstset{
    basicstyle=\ttfamily,
    mathescape
}
\usepackage{wasysym}
\usepackage{amsthm}
\usepackage{amssymb}
\usepackage{graphicx}
\usepackage{pdfpages}
\usepackage{float}
\usepackage{fancyhdr}
\usepackage[ruled,vlined]{algorithm2e}
\usepackage{xcolor}
\usepackage{mathtools}
\usepackage{amstext} % for \text macro
\usepackage{array}   % for \newcolumntype macro
% \usepackage{algorithm}
% \usepackage{algpseudocode}
% \usepackage[full]{complexity}
\newcolumntype{L}{>{$}l<{$}} % math-mode version of "l" column type
\DeclarePairedDelimiter{\ceil}{\lceil}{\rceil}

% \usepackage{hyperref}
% \hypersetup{
%     colorlinks=true,
%     linkcolor=blue,
%     filecolor=magenta,
%     urlcolor=cyan,
% }
\usepackage{CSTheoryToolkitCMUStyle}
\newcommand{\bigzero}{\mbox{\normalfont\Large\bfseries 0}}
\newcommand{\bigone}{\mbox{\normalfont\Large\bfseries 1}}
\newcommand{\mM}{\mathbf{M}}
\newcommand{\mA}{\mathbf{A}}
\newcommand{\mS}{\mathbf{S}}
\newcommand{\rvline}{\hspace*{-\arraycolsep}\vline\hspace*{-\arraycolsep}}

\definecolor{codegreen}{rgb}{0,0.6,0}
\definecolor{codegray}{rgb}{0.5,0.5,0.5}
\definecolor{codepurple}{rgb}{0.58,0,0.82}
\definecolor{backcolour}{rgb}{0.95,0.95,0.92}

\lstdefinestyle{mystyle}{
    backgroundcolor=\color{backcolour},
    commentstyle=\color{codegreen},
    keywordstyle=\color{magenta},
    numberstyle=\tiny\color{codegray},
    stringstyle=\color{codepurple},
    basicstyle=\ttfamily\footnotesize,
    breakatwhitespace=false,
    breaklines=true,
    captionpos=b,
    keepspaces=true,
    numbers=left,
    numbersep=5pt,
    showspaces=false,
    showstringspaces=false,
    showtabs=false,
    tabsize=2
}

\lstset{style=mystyle}

\oddsidemargin0cm
\topmargin-2cm     %I recommend adding these three lines to increase the
\textwidth16.5cm   %amount of usable space on the page (and save trees)
\textheight23.5cm

\newcommand{\question}[2] {\vspace{.25in} \hrule\vspace{0.5em}
    \noindent{\bf #1: #2} \vspace{0.5em}
    \hrule \vspace{.10in}}
\renewcommand{\part}[1] {\vspace{.10in} {\bf (#1)}}

\newcommand{\myname}{Fan Pu Zeng}
\newcommand{\myandrew}{fzeng@andrew.cmu.edu}
\newcommand{\myhwnum}{1}

\setlength{\parindent}{0pt} \setlength{\parskip}{5pt plus 1pt}

\pagestyle{fancyplain}
\lhead{\fancyplain{}{\textbf{HW\myhwnum}}}      % Note the different brackets!
\rhead{\fancyplain{}{\myname\\ \myandrew}}
\chead{\fancyplain{}{Scratch Pad}}

\newcommand{\clearthin}{\usefont{\encodingdefault}{ClearSans-TLF}{thin}{n}}
\newcommand{\definequantifier}[3]{%1 = command, #2 = h or v, #3 = letter
    \if #2h%
        \DeclareRobustCommand{#1}{\scalebox{-1}[1]{\text{\clearthin#3}}}%
    \else
        \DeclareRobustCommand{#1}{\raisebox{\depth}{\scalebox{1}[-1]{\text{\clearthin#3}}}}%
    \fi
}

\DeclareMathOperator*{\argmax}{arg\,max}
\DeclareMathOperator*{\argmin}{arg\,min}

\allowdisplaybreaks

% Add rulers for algorithm2e
\RestyleAlgo{ruled}
\SetKwComment{Comment}{/* }{ */}

\newcommand{\nnz}[1]{\mathtt{nnz}(#1)}


\begin{document}

% \includepdf[pages=-]{hw10_handwritten.pdf}

\medskip                        % Skip a "medium" amount of space
% (latex determines what medium is)
% Also try: \bigskip, \littleskip

\thispagestyle{plain}
\begin{center}                  % Center the following lines
    {\Large 15-859 Algorithms for Big Data Assignment \myhwnum} \\
    \myname \\
    \myandrew \\
\end{center}

\question{1}{Scratcy Scratch}

\begin{proof}
    Hello!
    \begin{align*}
        \parens{a} \\
        \braces{b} \\
        \bracks{c} \\
    \end{align*}
\end{proof}

\begin{align*}
    \abs{ \Pr \left[ S \leq u \right] - \Pr \left[ Z \leq u \right] }
    \leq \mbox{const} \cdot \beta,
\end{align*}
where the exact constant depends on the proof, with the best known constant
being $.5600$, and
$\beta = \sum\limits_{i=1}^n \E \left[ \abs{X_i}^3 \right]$.

\begin{align*}
    Z_n = \frac{\sqrt{n} \left( \overline{X}_n - \mu \right)}{\sigma}
\end{align*}

\begin{align*}
    \E \left[ e^{tX} \right]
\end{align*}

\begin{align*}
    M(0) & = \E \left[ e^{tX} \right] \Big|_{t=0} \\
         & = \E \left[ 1 \right]                  \\
         & = 1
\end{align*}

\begin{align*}
    M(t) & = \E \left[ e^{tX} \right]
         & = \E \left[ 1 \right]      \\
         & = 1
\end{align*}

\begin{align*}
    M^{(k)}(0) & = \frac{d^k}{dt^k} \E \left[ e^{tX} \right] \Big|_{t=0} \\
               & = \frac{d^k}{dt^k} \E \left[ e^{tX} \right] \Big|_{t=0} \\
\end{align*}

\begin{align*}
    M^{(k)}(t) & = \frac{d^k}{dt^k} \E \left[ e^{tX} \right]                      \\
               & = \frac{d}{dt} \E \left[ X^{k-1} e^{tX} \right] & \text{(by IH)} \\
               & = \frac{d}{dt} \int f(x) x^{k-1} e^{tx} \; dx                    \\
               & = \int \frac{d}{dt} f(x) x^{k-1} e^{tx} \; dx                    \\
               & = \int f(x) x^{k} e^{tx} \; dx                                   \\
               & = \E \left[ X^{k} e^{tX} \right].
\end{align*}

\begin{align*}
    % A_i = \frac{X_i - \mu}{\sigma}
    Z_n = \frac{1}{\sqrt{n}} \sum\limits_{i=1}^n A_i
\end{align*}

\begin{align*}
    \frac{1}{\sqrt{n}} \sum\limits_{i=1}^n A_i
     & = \frac{1}{\sqrt{n}} \sum\limits_{i=1}^n \frac{X_i - \mu}{\sigma} \\
     & = \sqrt{n} \sum\limits_{i=1}^n \frac{X_i - \mu}{ n \sigma}        \\
     & = \sqrt{n} \frac{\overline{X}_n - \mu}{ \sigma}                   \\
     & = Z_n.
\end{align*}

\begin{align*}
    M_{Z_n}(t) & = \E \left[ e^{t Z_n} \right]                                                                                                   \\
               & = \E \left[ \exp\left(t \frac{1}{\sqrt{n}} \sum\limits_{i=1}^n A_i \right) \right] & \text{(by equivalent definition of $Z_n$)} \\
               & = \prod_{i=1}^n \E \left[ \exp\left( \frac{t}{\sqrt{n}} A_i \right) \right]        & \text{(by independence of $A_i$'s)}        \\
               & = \prod_{i=1}^n M_{A_i}(t/\sqrt{n})                                                & \text{(definition of $M_{A_i}$)}           \\
               & = M_{A_i}(t/\sqrt{n} )^n .
\end{align*}

\begin{align*}
    f(x) = \sum\limits_{n=0}^\infty \frac{f^{(n)(a)}}{n!}(x-a)^n
\end{align*}

\begin{align*}
    f(x) = \sum\limits_{n=0}^\infty \frac{f^{(n)(a)}}{n!}(x-a)^n
\end{align*}

Our first three moments are
\begin{align*}
    M_{A_i}(0)                 & = \E \left[ e^{t A_i} \right] \Big|_{t=0}                                                                                                                 \\
                               & = \E \left[ 1 \right]                                                                                                                                     \\
                               & = 1,                                                                                                                                                      \\
    M_{A_i}^\prime(0)          & = \E \left[ A_i \right]                                                       & \text{(by the $k$th moment property proved previously)}                   \\
                               & = 0,                                                                                                                                                      \\
    M_{A_i}^{\prime \prime}(0) & = \E \left[ A_i^2 \right]                                                     & \text{(by the $k$th moment property proved previously)}                   \\
                               & = \E \left[ A_i^2 \right] - \E \left[ A_i \right]^2 + \E \left[ A_i \right]^2                                                                             \\
                               & = \Var(A_i) + \E \left[ A_i \right]^2                                         & \text{($\Var(A_i) = \E \left[ A_i^2 \right] - \E \left[ A_i \right]^2 $)} \\
                               & = 1 + 0                                                                                                                                                   \\
                               & = 1.
\end{align*}

\begin{align*}
    M_{A_i}(t/\sqrt{n}) & \approx M_{A_i}(0) + M_{A_i}^\prime(0) + M_{A_i}^{\prime \prime}(0) \frac{t^2}{2n} \\
                        & = 1 + 0 + \frac{t^2}{2n}                                                           \\
                        & = 1 + \frac{t^2}{2n}                                                               \\
\end{align*}

\begin{align*}
    M_{Z_n}(t) & = M_{A_i}(t/\sqrt{n})^n                                                                                       \\
               & \approx \left( 1 + \frac{t^2}{2n}  \right)^n                                                                  \\
               & \to e^{t^2/2}.                               & \text{(by identity $\lim_{n \to \infty} (1 + x/n)^n \to e^x$)}
\end{align*}

\begin{align*}
    M_{Z} & = \E \left[ e^{tZ} \right]                                                                                                \\
          & = \int f_Z(x) e^{tx} \; dx                                                                                                \\
          & = \int \frac{1}{\sqrt{2 \pi}} e^{-\frac{1}{2}x^2} e^{tx} \; dx                 & \text{(subst. pdf of standard Gaussian)} \\
          & = \int \frac{1}{\sqrt{2 \pi}} e^{-\frac{1}{2}x^2 + tx} \; dx                                                              \\
          & = \int \frac{1}{\sqrt{2 \pi}} e^{-\frac{1}{2}(x - t)^2 + \frac{1}{2}t^2} \; dx & \text{(completing the square)}           \\
          & = e^{\frac{1}{2}t^2} \int \frac{1}{\sqrt{2 \pi}} e^{-\frac{1}{2}(x - t)^2 } \; dx & \text{($e^{\frac{1}{2}t^2}$ does not depend on $x$)}           \\
          & = e^{\frac{1}{2}t^2} \cdot  1 \\
          & = e^{\frac{1}{2}t^2},
\end{align*}
where the second last step comes from the fact that
$$\frac{1}{\sqrt{2 \pi}} e^{-\frac{1}{2}(x - t)^2 }$$ is a probability distribution of a Gaussian with mean $$t$$ and variance 1, 
and therefore the integral integrates to 1.

\beta = 1/\poly(n)

\end{document}
